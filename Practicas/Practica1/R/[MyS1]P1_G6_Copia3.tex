% Options for packages loaded elsewhere
\PassOptionsToPackage{unicode}{hyperref}
\PassOptionsToPackage{hyphens}{url}
%
\documentclass[
]{article}
\usepackage{amsmath,amssymb}
\usepackage{iftex}
\ifPDFTeX
  \usepackage[T1]{fontenc}
  \usepackage[utf8]{inputenc}
  \usepackage{textcomp} % provide euro and other symbols
\else % if luatex or xetex
  \usepackage{unicode-math} % this also loads fontspec
  \defaultfontfeatures{Scale=MatchLowercase}
  \defaultfontfeatures[\rmfamily]{Ligatures=TeX,Scale=1}
\fi
\usepackage{lmodern}
\ifPDFTeX\else
  % xetex/luatex font selection
\fi
% Use upquote if available, for straight quotes in verbatim environments
\IfFileExists{upquote.sty}{\usepackage{upquote}}{}
\IfFileExists{microtype.sty}{% use microtype if available
  \usepackage[]{microtype}
  \UseMicrotypeSet[protrusion]{basicmath} % disable protrusion for tt fonts
}{}
\makeatletter
\@ifundefined{KOMAClassName}{% if non-KOMA class
  \IfFileExists{parskip.sty}{%
    \usepackage{parskip}
  }{% else
    \setlength{\parindent}{0pt}
    \setlength{\parskip}{6pt plus 2pt minus 1pt}}
}{% if KOMA class
  \KOMAoptions{parskip=half}}
\makeatother
\usepackage{xcolor}
\usepackage[margin=1in]{geometry}
\usepackage{color}
\usepackage{fancyvrb}
\newcommand{\VerbBar}{|}
\newcommand{\VERB}{\Verb[commandchars=\\\{\}]}
\DefineVerbatimEnvironment{Highlighting}{Verbatim}{commandchars=\\\{\}}
% Add ',fontsize=\small' for more characters per line
\usepackage{framed}
\definecolor{shadecolor}{RGB}{248,248,248}
\newenvironment{Shaded}{\begin{snugshade}}{\end{snugshade}}
\newcommand{\AlertTok}[1]{\textcolor[rgb]{0.94,0.16,0.16}{#1}}
\newcommand{\AnnotationTok}[1]{\textcolor[rgb]{0.56,0.35,0.01}{\textbf{\textit{#1}}}}
\newcommand{\AttributeTok}[1]{\textcolor[rgb]{0.13,0.29,0.53}{#1}}
\newcommand{\BaseNTok}[1]{\textcolor[rgb]{0.00,0.00,0.81}{#1}}
\newcommand{\BuiltInTok}[1]{#1}
\newcommand{\CharTok}[1]{\textcolor[rgb]{0.31,0.60,0.02}{#1}}
\newcommand{\CommentTok}[1]{\textcolor[rgb]{0.56,0.35,0.01}{\textit{#1}}}
\newcommand{\CommentVarTok}[1]{\textcolor[rgb]{0.56,0.35,0.01}{\textbf{\textit{#1}}}}
\newcommand{\ConstantTok}[1]{\textcolor[rgb]{0.56,0.35,0.01}{#1}}
\newcommand{\ControlFlowTok}[1]{\textcolor[rgb]{0.13,0.29,0.53}{\textbf{#1}}}
\newcommand{\DataTypeTok}[1]{\textcolor[rgb]{0.13,0.29,0.53}{#1}}
\newcommand{\DecValTok}[1]{\textcolor[rgb]{0.00,0.00,0.81}{#1}}
\newcommand{\DocumentationTok}[1]{\textcolor[rgb]{0.56,0.35,0.01}{\textbf{\textit{#1}}}}
\newcommand{\ErrorTok}[1]{\textcolor[rgb]{0.64,0.00,0.00}{\textbf{#1}}}
\newcommand{\ExtensionTok}[1]{#1}
\newcommand{\FloatTok}[1]{\textcolor[rgb]{0.00,0.00,0.81}{#1}}
\newcommand{\FunctionTok}[1]{\textcolor[rgb]{0.13,0.29,0.53}{\textbf{#1}}}
\newcommand{\ImportTok}[1]{#1}
\newcommand{\InformationTok}[1]{\textcolor[rgb]{0.56,0.35,0.01}{\textbf{\textit{#1}}}}
\newcommand{\KeywordTok}[1]{\textcolor[rgb]{0.13,0.29,0.53}{\textbf{#1}}}
\newcommand{\NormalTok}[1]{#1}
\newcommand{\OperatorTok}[1]{\textcolor[rgb]{0.81,0.36,0.00}{\textbf{#1}}}
\newcommand{\OtherTok}[1]{\textcolor[rgb]{0.56,0.35,0.01}{#1}}
\newcommand{\PreprocessorTok}[1]{\textcolor[rgb]{0.56,0.35,0.01}{\textit{#1}}}
\newcommand{\RegionMarkerTok}[1]{#1}
\newcommand{\SpecialCharTok}[1]{\textcolor[rgb]{0.81,0.36,0.00}{\textbf{#1}}}
\newcommand{\SpecialStringTok}[1]{\textcolor[rgb]{0.31,0.60,0.02}{#1}}
\newcommand{\StringTok}[1]{\textcolor[rgb]{0.31,0.60,0.02}{#1}}
\newcommand{\VariableTok}[1]{\textcolor[rgb]{0.00,0.00,0.00}{#1}}
\newcommand{\VerbatimStringTok}[1]{\textcolor[rgb]{0.31,0.60,0.02}{#1}}
\newcommand{\WarningTok}[1]{\textcolor[rgb]{0.56,0.35,0.01}{\textbf{\textit{#1}}}}
\usepackage{graphicx}
\makeatletter
\newsavebox\pandoc@box
\newcommand*\pandocbounded[1]{% scales image to fit in text height/width
  \sbox\pandoc@box{#1}%
  \Gscale@div\@tempa{\textheight}{\dimexpr\ht\pandoc@box+\dp\pandoc@box\relax}%
  \Gscale@div\@tempb{\linewidth}{\wd\pandoc@box}%
  \ifdim\@tempb\p@<\@tempa\p@\let\@tempa\@tempb\fi% select the smaller of both
  \ifdim\@tempa\p@<\p@\scalebox{\@tempa}{\usebox\pandoc@box}%
  \else\usebox{\pandoc@box}%
  \fi%
}
% Set default figure placement to htbp
\def\fps@figure{htbp}
\makeatother
\setlength{\emergencystretch}{3em} % prevent overfull lines
\providecommand{\tightlist}{%
  \setlength{\itemsep}{0pt}\setlength{\parskip}{0pt}}
\setcounter{secnumdepth}{-\maxdimen} % remove section numbering
\usepackage{booktabs}
\usepackage{longtable}
\usepackage{array}
\usepackage{multirow}
\usepackage{wrapfig}
\usepackage{float}
\usepackage{colortbl}
\usepackage{pdflscape}
\usepackage{tabu}
\usepackage{threeparttable}
\usepackage{threeparttablex}
\usepackage[normalem]{ulem}
\usepackage{makecell}
\usepackage{xcolor}
\usepackage{bookmark}
\IfFileExists{xurl.sty}{\usepackage{xurl}}{} % add URL line breaks if available
\urlstyle{same}
\hypersetup{
  pdftitle={Practica 1 - G8},
  hidelinks,
  pdfcreator={LaTeX via pandoc}}

\title{Practica 1 - G8}
\author{}
\date{\vspace{-2.5em}}

\begin{document}
\maketitle

Smart Packaging es una solución innovadora que demuestra el potencial de
aplicar tecnología avanzada a los procesos logísticos de empaque. Sin
embargo, los inversores de la empresa han expresado preocupación por el
incremento constante en los costos energéticos. Uno de los inversores
obtuvo información detallada sobre el consumo energético y la exportó en
el archivo POE\_2023.xlsx, el cual contiene registros del consumo medido
en MWh. Este inversor considera que dicha información es clave para
evaluar la rentabilidad del negocio. Con base en la construcción de un
modelo de análisis, se desea responder a las siguientes preguntas:

\section{1. ¿Cuál es el costo actual del consumo energético,
considerando que los robots consumen actualmente 0.2 MWh y operan bajo
el horario laboral vigente (08:00 a 20:00
horas)?}\label{cuuxe1l-es-el-costo-actual-del-consumo-energuxe9tico-considerando-que-los-robots-consumen-actualmente-0.2-mwh-y-operan-bajo-el-horario-laboral-vigente-0800-a-2000-horas}

\begin{Shaded}
\begin{Highlighting}[]
\CommentTok{\# Cargar las librerías}
\FunctionTok{library}\NormalTok{(readxl)    }
\FunctionTok{library}\NormalTok{(dplyr)     }
\end{Highlighting}
\end{Shaded}

\begin{verbatim}
## 
## Adjuntando el paquete: 'dplyr'
\end{verbatim}

\begin{verbatim}
## The following objects are masked from 'package:stats':
## 
##     filter, lag
\end{verbatim}

\begin{verbatim}
## The following objects are masked from 'package:base':
## 
##     intersect, setdiff, setequal, union
\end{verbatim}

\begin{Shaded}
\begin{Highlighting}[]
\FunctionTok{library}\NormalTok{(tidyr)     }
\FunctionTok{library}\NormalTok{(lubridate) }
\end{Highlighting}
\end{Shaded}

\begin{verbatim}
## 
## Adjuntando el paquete: 'lubridate'
\end{verbatim}

\begin{verbatim}
## The following objects are masked from 'package:base':
## 
##     date, intersect, setdiff, union
\end{verbatim}

\begin{Shaded}
\begin{Highlighting}[]
\FunctionTok{library}\NormalTok{(ggplot2) }

\NormalTok{ARCHIVO\_EXCEL }\OtherTok{\textless{}{-}} \StringTok{"POE\_2023.xlsx"}

\CommentTok{\# horario laboral: de 8:00 AM a 8:00 PM (20:00 horas)}
\NormalTok{HORARIO\_LABORAL }\OtherTok{\textless{}{-}} \DecValTok{8}\SpecialCharTok{:}\DecValTok{20} 

\CommentTok{\# días laborales promedio por mes}
\NormalTok{DIAS\_LABORALES\_MES }\OtherTok{\textless{}{-}} \DecValTok{30}

\CommentTok{\# Parámetros del problema}
\NormalTok{CONSUMO\_ACTUAL }\OtherTok{\textless{}{-}} \FloatTok{0.2}
\NormalTok{HORAS\_ACTUALES }\OtherTok{\textless{}{-}} \FunctionTok{length}\NormalTok{(HORARIO\_LABORAL) }\CommentTok{\#las horas que se trabajan al dia}

\NormalTok{MESES }\OtherTok{\textless{}{-}} \FunctionTok{c}\NormalTok{(}\StringTok{"ENERO"}\NormalTok{, }\StringTok{"FEBRERO"}\NormalTok{, }\StringTok{"MARZO"}\NormalTok{, }\StringTok{"ABRIL"}\NormalTok{, }\StringTok{"MAYO"}\NormalTok{, }\StringTok{"JUNIO"}\NormalTok{,}
           \StringTok{"JULIO"}\NormalTok{, }\StringTok{"AGOSTO"}\NormalTok{, }\StringTok{"SEPTIEMBRE"}\NormalTok{, }\StringTok{"OCTUBRE"}\NormalTok{, }\StringTok{"NOVIEMBRE"}\NormalTok{, }\StringTok{"DICIEMBRE"}\NormalTok{)}
\end{Highlighting}
\end{Shaded}

\subsection{Funcion para procesar la información del
excel}\label{funcion-para-procesar-la-informaciuxf3n-del-excel}

\begin{Shaded}
\begin{Highlighting}[]
\NormalTok{procesar\_datos\_energia }\OtherTok{\textless{}{-}} \ControlFlowTok{function}\NormalTok{(archivo\_excel) \{}
  
  \ControlFlowTok{if}\NormalTok{ (}\SpecialCharTok{!}\FunctionTok{file.exists}\NormalTok{(archivo\_excel)) \{}
    \FunctionTok{stop}\NormalTok{(}\FunctionTok{paste}\NormalTok{(}\StringTok{"Error: No se encuentra el archivo"}\NormalTok{, archivo\_excel))}
\NormalTok{  \}}
  
  \CommentTok{\#almacenar todos los datos procesados}
\NormalTok{  datos\_completos }\OtherTok{\textless{}{-}} \FunctionTok{data.frame}\NormalTok{()}
  
  \CommentTok{\# Procesar cada mes del año}
  \ControlFlowTok{for}\NormalTok{ (i }\ControlFlowTok{in} \DecValTok{1}\SpecialCharTok{:}\FunctionTok{length}\NormalTok{(MESES)) \{}
\NormalTok{    mes }\OtherTok{\textless{}{-}}\NormalTok{ MESES[i]}
    
    \FunctionTok{tryCatch}\NormalTok{(\{}
      \CommentTok{\# Leer la hoja correspondiente al mes actual del archivo Excel}
\NormalTok{      datos\_raw }\OtherTok{\textless{}{-}} \FunctionTok{read\_excel}\NormalTok{(archivo\_excel, }\AttributeTok{sheet =}\NormalTok{ mes, }\AttributeTok{col\_names =} \ConstantTok{FALSE}\NormalTok{, }\AttributeTok{col\_types =} \StringTok{"text"}\NormalTok{)}
      
      \CommentTok{\# Buscar en qué fila está la palabra "Hora" porque ahi van a iniciar los datos}
\NormalTok{      fila\_hora }\OtherTok{\textless{}{-}} \FunctionTok{which}\NormalTok{(datos\_raw[[}\DecValTok{1}\NormalTok{]] }\SpecialCharTok{==} \StringTok{"Hora"}\NormalTok{, }\AttributeTok{arr.ind =} \ConstantTok{TRUE}\NormalTok{)}
      
      \ControlFlowTok{if}\NormalTok{ (}\FunctionTok{length}\NormalTok{(fila\_hora) }\SpecialCharTok{\textgreater{}} \DecValTok{0}\NormalTok{) \{}
        \CommentTok{\# Los datos reales empiezan en la fila de abajo}
\NormalTok{        inicio\_datos }\OtherTok{\textless{}{-}}\NormalTok{ fila\_hora[}\DecValTok{1}\NormalTok{] }\SpecialCharTok{+} \DecValTok{1}
        \CommentTok{\# Extraer solo las filas que contienen datos}
\NormalTok{        datos\_mes }\OtherTok{\textless{}{-}}\NormalTok{ datos\_raw[inicio\_datos}\SpecialCharTok{:}\FunctionTok{nrow}\NormalTok{(datos\_raw), ]}
        
        \CommentTok{\# Sacamos los encabezados para saber los dias del mes}
\NormalTok{        encabezados }\OtherTok{\textless{}{-}}\NormalTok{ datos\_raw[fila\_hora[}\DecValTok{1}\NormalTok{], ]}
\NormalTok{        dias\_mes }\OtherTok{\textless{}{-}} \FunctionTok{suppressWarnings}\NormalTok{(}\FunctionTok{as.numeric}\NormalTok{(}\FunctionTok{unlist}\NormalTok{(encabezados[}\DecValTok{3}\SpecialCharTok{:}\FunctionTok{ncol}\NormalTok{(encabezados)])))}
        \CommentTok{\# Filtrar solo días válidos (entre 1 y 31, sin valores NA)}
\NormalTok{        dias\_mes }\OtherTok{\textless{}{-}}\NormalTok{ dias\_mes[}\SpecialCharTok{!}\FunctionTok{is.na}\NormalTok{(dias\_mes) }\SpecialCharTok{\&}\NormalTok{ dias\_mes }\SpecialCharTok{\textgreater{}=} \DecValTok{1} \SpecialCharTok{\&}\NormalTok{ dias\_mes }\SpecialCharTok{\textless{}=} \DecValTok{31}\NormalTok{]}
        
\NormalTok{        registros\_mes }\OtherTok{\textless{}{-}} \DecValTok{0}
        
        \CommentTok{\# Procesar cada fila de datos que vendra siendo cada hora del dia}
        \ControlFlowTok{for}\NormalTok{ (fila }\ControlFlowTok{in} \DecValTok{1}\SpecialCharTok{:}\FunctionTok{nrow}\NormalTok{(datos\_mes)) \{}
\NormalTok{          primera\_col }\OtherTok{\textless{}{-}}\NormalTok{ datos\_mes[fila, }\DecValTok{1}\NormalTok{]}
          
          \CommentTok{\# Verificar que la celda no esté vacía}
          \ControlFlowTok{if}\NormalTok{ (}\SpecialCharTok{!}\FunctionTok{is.na}\NormalTok{(primera\_col) }\SpecialCharTok{\&\&} \SpecialCharTok{!}\FunctionTok{is.null}\NormalTok{(primera\_col) }\SpecialCharTok{\&\&}\NormalTok{ primera\_col }\SpecialCharTok{!=} \StringTok{""}\NormalTok{) \{}
            
            \CommentTok{\# VAlidamos que los datos vengan en formato de fraccion que es el formato que usa excel para las horas}
            \CommentTok{\# suppressWarnings evita mostrar advertencias si la conversión falla}
\NormalTok{            hora\_fraccion }\OtherTok{\textless{}{-}} \FunctionTok{suppressWarnings}\NormalTok{(}\FunctionTok{as.numeric}\NormalTok{(primera\_col))}
            
            \ControlFlowTok{if}\NormalTok{ (}\SpecialCharTok{!}\FunctionTok{is.na}\NormalTok{(hora\_fraccion) }\SpecialCharTok{\&\&} \FunctionTok{is.numeric}\NormalTok{(hora\_fraccion)) \{}
              \CommentTok{\# Convertir la fracción de día a hora del día (0{-}23) para que sea facil de entender}
\NormalTok{              hora }\OtherTok{\textless{}{-}} \FunctionTok{floor}\NormalTok{(hora\_fraccion }\SpecialCharTok{*} \DecValTok{24}\NormalTok{)}
              
              \ControlFlowTok{if}\NormalTok{ (}\SpecialCharTok{!}\FunctionTok{is.na}\NormalTok{(hora) }\SpecialCharTok{\&\&}\NormalTok{ hora }\SpecialCharTok{\textgreater{}=} \DecValTok{0} \SpecialCharTok{\&\&}\NormalTok{ hora }\SpecialCharTok{\textless{}=} \DecValTok{23}\NormalTok{) \{}
                
                \CommentTok{\# Procesar cada día del mes actual}
                \ControlFlowTok{for}\NormalTok{ (col }\ControlFlowTok{in} \DecValTok{1}\SpecialCharTok{:}\FunctionTok{length}\NormalTok{(dias\_mes)) \{}
\NormalTok{                  dia }\OtherTok{\textless{}{-}}\NormalTok{ dias\_mes[col]}
                  
                  \CommentTok{\# Obtener el precio de energía para esta hora y día específicos}
                  \CommentTok{\# col + 2 porque las primeras 2 columnas son hora, los días empiezan en la 3ra}
\NormalTok{                  precio\_raw }\OtherTok{\textless{}{-}}\NormalTok{ datos\_mes[fila, col }\SpecialCharTok{+} \DecValTok{2}\NormalTok{]}
\NormalTok{                  precio }\OtherTok{\textless{}{-}} \FunctionTok{suppressWarnings}\NormalTok{(}\FunctionTok{as.numeric}\NormalTok{(precio\_raw))}
                  
                  \ControlFlowTok{if}\NormalTok{ (}\SpecialCharTok{!}\FunctionTok{is.na}\NormalTok{(precio) }\SpecialCharTok{\&\&} \SpecialCharTok{!}\FunctionTok{is.na}\NormalTok{(dia) }\SpecialCharTok{\&\&}\NormalTok{ precio }\SpecialCharTok{\textgreater{}} \DecValTok{0}\NormalTok{) \{}
                    \CommentTok{\# Agregar este registro al data frame de datos completos}
\NormalTok{                    datos\_completos }\OtherTok{\textless{}{-}} \FunctionTok{rbind}\NormalTok{(datos\_completos, }\FunctionTok{data.frame}\NormalTok{(}
                      \AttributeTok{mes =}\NormalTok{ mes,              }
                      \AttributeTok{mes\_num =}\NormalTok{ i,            }
                      \AttributeTok{dia =}\NormalTok{ dia,              }
                      \AttributeTok{hora =}\NormalTok{ hora,            }
                      \AttributeTok{precio\_mwh =}\NormalTok{ precio,    }\CommentTok{\# Precio de energía en USD/MWh}
                      \AttributeTok{stringsAsFactors =} \ConstantTok{FALSE} 
\NormalTok{                    ))}
                    \CommentTok{\# Incrementar el contador de registros procesados para este mes}
\NormalTok{                    registros\_mes }\OtherTok{\textless{}{-}}\NormalTok{ registros\_mes }\SpecialCharTok{+} \DecValTok{1}
\NormalTok{                  \}}
\NormalTok{                \}}
\NormalTok{              \}}
\NormalTok{            \}}
\NormalTok{          \}}
\NormalTok{        \}}
        \FunctionTok{cat}\NormalTok{(registros\_mes, }\StringTok{"registros}\SpecialCharTok{\textbackslash{}n}\StringTok{"}\NormalTok{)}
\NormalTok{      \} }\ControlFlowTok{else}\NormalTok{ \{}
        \CommentTok{\# Si no se encontró la fila de "Hora"}
        \FunctionTok{cat}\NormalTok{(}\StringTok{"no se encontró fila \textquotesingle{}Hora\textquotesingle{}}\SpecialCharTok{\textbackslash{}n}\StringTok{"}\NormalTok{)}
\NormalTok{      \}}
\NormalTok{    \}, }\AttributeTok{error =} \ControlFlowTok{function}\NormalTok{(e) \{}
      \CommentTok{\# Si ocurre algún error durante el procesamiento, mostrar el mensaje}
      \FunctionTok{cat}\NormalTok{(}\StringTok{"Error:"}\NormalTok{, }\FunctionTok{conditionMessage}\NormalTok{(e), }\StringTok{"}\SpecialCharTok{\textbackslash{}n}\StringTok{"}\NormalTok{)}
\NormalTok{    \})}
\NormalTok{  \}}
  
  \CommentTok{\# Mostrar el total de registros procesados de todos los meses}
  \FunctionTok{cat}\NormalTok{(}\StringTok{"Total de registros procesados:"}\NormalTok{, }\FunctionTok{nrow}\NormalTok{(datos\_completos), }\StringTok{"}\SpecialCharTok{\textbackslash{}n}\StringTok{"}\NormalTok{)}
  \FunctionTok{return}\NormalTok{(datos\_completos)}
\NormalTok{\}}
\end{Highlighting}
\end{Shaded}

\subsubsection{Procesar datos}\label{procesar-datos}

\begin{Shaded}
\begin{Highlighting}[]
\CommentTok{\# Ejecutar la función para procesar todos los datos del archivo Excel}
\NormalTok{datos }\OtherTok{\textless{}{-}} \FunctionTok{procesar\_datos\_energia}\NormalTok{(ARCHIVO\_EXCEL)}
\end{Highlighting}
\end{Shaded}

\begin{verbatim}
## New names:
## * `` -> `...1`
## * `` -> `...2`
## * `` -> `...3`
## * `` -> `...4`
## * `` -> `...5`
## * `` -> `...6`
## * `` -> `...7`
## * `` -> `...8`
## * `` -> `...9`
## * `` -> `...10`
## * `` -> `...11`
## * `` -> `...12`
## * `` -> `...13`
## * `` -> `...14`
## * `` -> `...15`
## * `` -> `...16`
## * `` -> `...17`
## * `` -> `...18`
## * `` -> `...19`
## * `` -> `...20`
## * `` -> `...21`
## * `` -> `...22`
## * `` -> `...23`
## * `` -> `...24`
## * `` -> `...25`
## * `` -> `...26`
## * `` -> `...27`
## * `` -> `...28`
## * `` -> `...29`
## * `` -> `...30`
## * `` -> `...31`
## * `` -> `...32`
## * `` -> `...33`
\end{verbatim}

\begin{verbatim}
## 744 registros
\end{verbatim}

\begin{verbatim}
## New names:
## * `` -> `...1`
## * `` -> `...2`
## * `` -> `...3`
## * `` -> `...4`
## * `` -> `...5`
## * `` -> `...6`
## * `` -> `...7`
## * `` -> `...8`
## * `` -> `...9`
## * `` -> `...10`
## * `` -> `...11`
## * `` -> `...12`
## * `` -> `...13`
## * `` -> `...14`
## * `` -> `...15`
## * `` -> `...16`
## * `` -> `...17`
## * `` -> `...18`
## * `` -> `...19`
## * `` -> `...20`
## * `` -> `...21`
## * `` -> `...22`
## * `` -> `...23`
## * `` -> `...24`
## * `` -> `...25`
## * `` -> `...26`
## * `` -> `...27`
## * `` -> `...28`
## * `` -> `...29`
## * `` -> `...30`
## * `` -> `...31`
## * `` -> `...32`
## * `` -> `...33`
\end{verbatim}

\begin{verbatim}
## 672 registros
\end{verbatim}

\begin{verbatim}
## New names:
## * `` -> `...1`
## * `` -> `...2`
## * `` -> `...3`
## * `` -> `...4`
## * `` -> `...5`
## * `` -> `...6`
## * `` -> `...7`
## * `` -> `...8`
## * `` -> `...9`
## * `` -> `...10`
## * `` -> `...11`
## * `` -> `...12`
## * `` -> `...13`
## * `` -> `...14`
## * `` -> `...15`
## * `` -> `...16`
## * `` -> `...17`
## * `` -> `...18`
## * `` -> `...19`
## * `` -> `...20`
## * `` -> `...21`
## * `` -> `...22`
## * `` -> `...23`
## * `` -> `...24`
## * `` -> `...25`
## * `` -> `...26`
## * `` -> `...27`
## * `` -> `...28`
## * `` -> `...29`
## * `` -> `...30`
## * `` -> `...31`
## * `` -> `...32`
## * `` -> `...33`
\end{verbatim}

\begin{verbatim}
## 744 registros
\end{verbatim}

\begin{verbatim}
## New names:
## * `` -> `...1`
## * `` -> `...2`
## * `` -> `...3`
## * `` -> `...4`
## * `` -> `...5`
## * `` -> `...6`
## * `` -> `...7`
## * `` -> `...8`
## * `` -> `...9`
## * `` -> `...10`
## * `` -> `...11`
## * `` -> `...12`
## * `` -> `...13`
## * `` -> `...14`
## * `` -> `...15`
## * `` -> `...16`
## * `` -> `...17`
## * `` -> `...18`
## * `` -> `...19`
## * `` -> `...20`
## * `` -> `...21`
## * `` -> `...22`
## * `` -> `...23`
## * `` -> `...24`
## * `` -> `...25`
## * `` -> `...26`
## * `` -> `...27`
## * `` -> `...28`
## * `` -> `...29`
## * `` -> `...30`
## * `` -> `...31`
## * `` -> `...32`
## * `` -> `...33`
\end{verbatim}

\begin{verbatim}
## 720 registros
\end{verbatim}

\begin{verbatim}
## New names:
## * `` -> `...1`
## * `` -> `...2`
## * `` -> `...3`
## * `` -> `...4`
## * `` -> `...5`
## * `` -> `...6`
## * `` -> `...7`
## * `` -> `...8`
## * `` -> `...9`
## * `` -> `...10`
## * `` -> `...11`
## * `` -> `...12`
## * `` -> `...13`
## * `` -> `...14`
## * `` -> `...15`
## * `` -> `...16`
## * `` -> `...17`
## * `` -> `...18`
## * `` -> `...19`
## * `` -> `...20`
## * `` -> `...21`
## * `` -> `...22`
## * `` -> `...23`
## * `` -> `...24`
## * `` -> `...25`
## * `` -> `...26`
## * `` -> `...27`
## * `` -> `...28`
## * `` -> `...29`
## * `` -> `...30`
## * `` -> `...31`
## * `` -> `...32`
## * `` -> `...33`
\end{verbatim}

\begin{verbatim}
## 744 registros
\end{verbatim}

\begin{verbatim}
## New names:
## * `` -> `...1`
## * `` -> `...2`
## * `` -> `...3`
## * `` -> `...4`
## * `` -> `...5`
## * `` -> `...6`
## * `` -> `...7`
## * `` -> `...8`
## * `` -> `...9`
## * `` -> `...10`
## * `` -> `...11`
## * `` -> `...12`
## * `` -> `...13`
## * `` -> `...14`
## * `` -> `...15`
## * `` -> `...16`
## * `` -> `...17`
## * `` -> `...18`
## * `` -> `...19`
## * `` -> `...20`
## * `` -> `...21`
## * `` -> `...22`
## * `` -> `...23`
## * `` -> `...24`
## * `` -> `...25`
## * `` -> `...26`
## * `` -> `...27`
## * `` -> `...28`
## * `` -> `...29`
## * `` -> `...30`
## * `` -> `...31`
## * `` -> `...32`
## * `` -> `...33`
\end{verbatim}

\begin{verbatim}
## 720 registros
\end{verbatim}

\begin{verbatim}
## New names:
## * `` -> `...1`
## * `` -> `...2`
## * `` -> `...3`
## * `` -> `...4`
## * `` -> `...5`
## * `` -> `...6`
## * `` -> `...7`
## * `` -> `...8`
## * `` -> `...9`
## * `` -> `...10`
## * `` -> `...11`
## * `` -> `...12`
## * `` -> `...13`
## * `` -> `...14`
## * `` -> `...15`
## * `` -> `...16`
## * `` -> `...17`
## * `` -> `...18`
## * `` -> `...19`
## * `` -> `...20`
## * `` -> `...21`
## * `` -> `...22`
## * `` -> `...23`
## * `` -> `...24`
## * `` -> `...25`
## * `` -> `...26`
## * `` -> `...27`
## * `` -> `...28`
## * `` -> `...29`
## * `` -> `...30`
## * `` -> `...31`
## * `` -> `...32`
## * `` -> `...33`
\end{verbatim}

\begin{verbatim}
## 744 registros
\end{verbatim}

\begin{verbatim}
## New names:
## * `` -> `...1`
## * `` -> `...2`
## * `` -> `...3`
## * `` -> `...4`
## * `` -> `...5`
## * `` -> `...6`
## * `` -> `...7`
## * `` -> `...8`
## * `` -> `...9`
## * `` -> `...10`
## * `` -> `...11`
## * `` -> `...12`
## * `` -> `...13`
## * `` -> `...14`
## * `` -> `...15`
## * `` -> `...16`
## * `` -> `...17`
## * `` -> `...18`
## * `` -> `...19`
## * `` -> `...20`
## * `` -> `...21`
## * `` -> `...22`
## * `` -> `...23`
## * `` -> `...24`
## * `` -> `...25`
## * `` -> `...26`
## * `` -> `...27`
## * `` -> `...28`
## * `` -> `...29`
## * `` -> `...30`
## * `` -> `...31`
## * `` -> `...32`
## * `` -> `...33`
\end{verbatim}

\begin{verbatim}
## 744 registros
\end{verbatim}

\begin{verbatim}
## New names:
## * `` -> `...1`
## * `` -> `...2`
## * `` -> `...3`
## * `` -> `...4`
## * `` -> `...5`
## * `` -> `...6`
## * `` -> `...7`
## * `` -> `...8`
## * `` -> `...9`
## * `` -> `...10`
## * `` -> `...11`
## * `` -> `...12`
## * `` -> `...13`
## * `` -> `...14`
## * `` -> `...15`
## * `` -> `...16`
## * `` -> `...17`
## * `` -> `...18`
## * `` -> `...19`
## * `` -> `...20`
## * `` -> `...21`
## * `` -> `...22`
## * `` -> `...23`
## * `` -> `...24`
## * `` -> `...25`
## * `` -> `...26`
## * `` -> `...27`
## * `` -> `...28`
## * `` -> `...29`
## * `` -> `...30`
## * `` -> `...31`
## * `` -> `...32`
## * `` -> `...33`
\end{verbatim}

\begin{verbatim}
## 720 registros
\end{verbatim}

\begin{verbatim}
## New names:
## * `` -> `...1`
## * `` -> `...2`
## * `` -> `...3`
## * `` -> `...4`
## * `` -> `...5`
## * `` -> `...6`
## * `` -> `...7`
## * `` -> `...8`
## * `` -> `...9`
## * `` -> `...10`
## * `` -> `...11`
## * `` -> `...12`
## * `` -> `...13`
## * `` -> `...14`
## * `` -> `...15`
## * `` -> `...16`
## * `` -> `...17`
## * `` -> `...18`
## * `` -> `...19`
## * `` -> `...20`
## * `` -> `...21`
## * `` -> `...22`
## * `` -> `...23`
## * `` -> `...24`
## * `` -> `...25`
## * `` -> `...26`
## * `` -> `...27`
## * `` -> `...28`
## * `` -> `...29`
## * `` -> `...30`
## * `` -> `...31`
## * `` -> `...32`
## * `` -> `...33`
\end{verbatim}

\begin{verbatim}
## 744 registros
\end{verbatim}

\begin{verbatim}
## New names:
## * `` -> `...1`
## * `` -> `...2`
## * `` -> `...3`
## * `` -> `...4`
## * `` -> `...5`
## * `` -> `...6`
## * `` -> `...7`
## * `` -> `...8`
## * `` -> `...9`
## * `` -> `...10`
## * `` -> `...11`
## * `` -> `...12`
## * `` -> `...13`
## * `` -> `...14`
## * `` -> `...15`
## * `` -> `...16`
## * `` -> `...17`
## * `` -> `...18`
## * `` -> `...19`
## * `` -> `...20`
## * `` -> `...21`
## * `` -> `...22`
## * `` -> `...23`
## * `` -> `...24`
## * `` -> `...25`
## * `` -> `...26`
## * `` -> `...27`
## * `` -> `...28`
## * `` -> `...29`
## * `` -> `...30`
## * `` -> `...31`
## * `` -> `...32`
## * `` -> `...33`
\end{verbatim}

\begin{verbatim}
## 720 registros
\end{verbatim}

\begin{verbatim}
## New names:
## * `` -> `...1`
## * `` -> `...2`
## * `` -> `...3`
## * `` -> `...4`
## * `` -> `...5`
## * `` -> `...6`
## * `` -> `...7`
## * `` -> `...8`
## * `` -> `...9`
## * `` -> `...10`
## * `` -> `...11`
## * `` -> `...12`
## * `` -> `...13`
## * `` -> `...14`
## * `` -> `...15`
## * `` -> `...16`
## * `` -> `...17`
## * `` -> `...18`
## * `` -> `...19`
## * `` -> `...20`
## * `` -> `...21`
## * `` -> `...22`
## * `` -> `...23`
## * `` -> `...24`
## * `` -> `...25`
## * `` -> `...26`
## * `` -> `...27`
## * `` -> `...28`
## * `` -> `...29`
## * `` -> `...30`
## * `` -> `...31`
## * `` -> `...32`
## * `` -> `...33`
\end{verbatim}

\begin{verbatim}
## 744 registros
## Total de registros procesados: 8760
\end{verbatim}

\begin{Shaded}
\begin{Highlighting}[]
\CommentTok{\# Filtrar los datos para incluir solo las horas del horario laboral (8:00 a 20:00)}
\NormalTok{datos\_laborales }\OtherTok{\textless{}{-}}\NormalTok{ datos }\SpecialCharTok{\%\textgreater{}\%}
  \FunctionTok{filter}\NormalTok{(hora }\SpecialCharTok{\%in\%}\NormalTok{ HORARIO\_LABORAL)}

\NormalTok{resumen\_datos }\OtherTok{\textless{}{-}}\NormalTok{ datos\_laborales }\SpecialCharTok{\%\textgreater{}\%}
  \FunctionTok{group\_by}\NormalTok{(mes) }\SpecialCharTok{\%\textgreater{}\%}  \CommentTok{\# Agrupar los datos por mes}
  \FunctionTok{summarise}\NormalTok{(}
    \AttributeTok{registros =} \FunctionTok{n}\NormalTok{(),  }
    \AttributeTok{precio\_min =} \FunctionTok{min}\NormalTok{(precio\_mwh, }\AttributeTok{na.rm =} \ConstantTok{TRUE}\NormalTok{),      }
    \AttributeTok{precio\_max =} \FunctionTok{max}\NormalTok{(precio\_mwh, }\AttributeTok{na.rm =} \ConstantTok{TRUE}\NormalTok{),      }
    \AttributeTok{precio\_promedio =} \FunctionTok{mean}\NormalTok{(precio\_mwh, }\AttributeTok{na.rm =} \ConstantTok{TRUE}\NormalTok{), }
    \AttributeTok{.groups =} \StringTok{\textquotesingle{}drop\textquotesingle{}}
\NormalTok{  )}
\end{Highlighting}
\end{Shaded}

\subsubsection{Calculos}\label{calculos}

\begin{Shaded}
\begin{Highlighting}[]
\CommentTok{\# Calcular los costos mensuales basados en el consumo actual}
\NormalTok{costos\_mensuales }\OtherTok{\textless{}{-}}\NormalTok{ datos\_laborales }\SpecialCharTok{\%\textgreater{}\%}
  \FunctionTok{group\_by}\NormalTok{(mes, mes\_num) }\SpecialCharTok{\%\textgreater{}\%}  \CommentTok{\# Agrupar por mes y número de mes}
  \FunctionTok{summarise}\NormalTok{(}
    \AttributeTok{precio\_promedio =} \FunctionTok{mean}\NormalTok{(precio\_mwh, }\AttributeTok{na.rm =} \ConstantTok{TRUE}\NormalTok{),}
    \AttributeTok{registros =} \FunctionTok{n}\NormalTok{(),  }
    \AttributeTok{.groups =} \StringTok{\textquotesingle{}drop\textquotesingle{}}  
\NormalTok{  ) }\SpecialCharTok{\%\textgreater{}\%}
  \FunctionTok{mutate}\NormalTok{(}
    \CommentTok{\# Precio promedio × Consumo por hora × Horas por día × Días por mes}
    \AttributeTok{costo\_mensual\_actual =}\NormalTok{ precio\_promedio }\SpecialCharTok{*}\NormalTok{ CONSUMO\_ACTUAL }\SpecialCharTok{*}\NormalTok{ HORAS\_ACTUALES }\SpecialCharTok{*}\NormalTok{ DIAS\_LABORALES\_MES}
\NormalTok{  ) }\SpecialCharTok{\%\textgreater{}\%}
  \FunctionTok{arrange}\NormalTok{(mes\_num)}

\CommentTok{\# Calcular el costo total anual sumando todos los costos mensuales}
\NormalTok{costo\_anual\_actual }\OtherTok{\textless{}{-}} \FunctionTok{sum}\NormalTok{(costos\_mensuales}\SpecialCharTok{$}\NormalTok{costo\_mensual\_actual)}
\end{Highlighting}
\end{Shaded}

\subsubsection{RESPUESTA 1}\label{respuesta-1}

\begin{Shaded}
\begin{Highlighting}[]
\FunctionTok{cat}\NormalTok{(}\StringTok{"}\SpecialCharTok{\textbackslash{}n}\StringTok{COSTO ACTUAL DEL CONSUMO ENERGÉTICO}\SpecialCharTok{\textbackslash{}n}\StringTok{"}\NormalTok{, }\StringTok{"Costo anual total:"}\NormalTok{, scales}\SpecialCharTok{::}\FunctionTok{dollar}\NormalTok{(costo\_anual\_actual), }\StringTok{"}\SpecialCharTok{\textbackslash{}n}\StringTok{"}\NormalTok{, }\StringTok{"Costo mensual promedio:"}\NormalTok{, scales}\SpecialCharTok{::}\FunctionTok{dollar}\NormalTok{(costo\_anual\_actual}\SpecialCharTok{/}\DecValTok{12}\NormalTok{), }\StringTok{"}\SpecialCharTok{\textbackslash{}n}\StringTok{"}\NormalTok{)}
\end{Highlighting}
\end{Shaded}

\begin{verbatim}
## 
## COSTO ACTUAL DEL CONSUMO ENERGÉTICO
##  Costo anual total: $112,476 
##  Costo mensual promedio: $9,373.04
\end{verbatim}

\section{2. Si se modifica el modelo para que los robots consuman 0.15
MWh, pero trabajen la mitad del tiempo, ¿sigue siendo rentable la
operación?}\label{si-se-modifica-el-modelo-para-que-los-robots-consuman-0.15-mwh-pero-trabajen-la-mitad-del-tiempo-sigue-siendo-rentable-la-operaciuxf3n}

\begin{Shaded}
\begin{Highlighting}[]
\NormalTok{CONSUMO\_NUEVO }\OtherTok{\textless{}{-}} \FloatTok{0.15}
\NormalTok{HORAS\_NUEVAS }\OtherTok{\textless{}{-}}\NormalTok{ HORAS\_ACTUALES }\SpecialCharTok{/} \DecValTok{2}
\end{Highlighting}
\end{Shaded}

\subsubsection{Calculos}\label{calculos-1}

\begin{Shaded}
\begin{Highlighting}[]
\NormalTok{costos\_mensuales\_nuevos }\OtherTok{\textless{}{-}}\NormalTok{ costos\_mensuales }\SpecialCharTok{\%\textgreater{}\%}
  \FunctionTok{mutate}\NormalTok{(}
  \AttributeTok{costo\_mensual\_nuevo =}\NormalTok{ precio\_promedio }\SpecialCharTok{*}\NormalTok{ CONSUMO\_NUEVO }\SpecialCharTok{*}\NormalTok{ HORAS\_NUEVAS }\SpecialCharTok{*}\NormalTok{ DIAS\_LABORALES\_MES,}
  \AttributeTok{ahorro\_mensual =}\NormalTok{ costo\_mensual\_actual }\SpecialCharTok{{-}}\NormalTok{ costo\_mensual\_nuevo,}
    \AttributeTok{porcentaje\_ahorro =}\NormalTok{ (ahorro\_mensual }\SpecialCharTok{/}\NormalTok{ costo\_mensual\_actual) }\SpecialCharTok{*} \DecValTok{100}
\NormalTok{  )}

\NormalTok{costo\_anual\_nuevo }\OtherTok{\textless{}{-}} \FunctionTok{sum}\NormalTok{(costos\_mensuales\_nuevos}\SpecialCharTok{$}\NormalTok{costo\_mensual\_nuevo)}
\NormalTok{ahorro\_anual }\OtherTok{\textless{}{-}}\NormalTok{ costo\_anual\_actual }\SpecialCharTok{{-}}\NormalTok{ costo\_anual\_nuevo}
\NormalTok{porcentaje\_ahorro\_anual }\OtherTok{\textless{}{-}}\NormalTok{ (ahorro\_anual }\SpecialCharTok{/}\NormalTok{ costo\_anual\_actual) }\SpecialCharTok{*} \DecValTok{100}
\end{Highlighting}
\end{Shaded}

\subsubsection{RESPUESTA 2}\label{respuesta-2}

\begin{Shaded}
\begin{Highlighting}[]
\FunctionTok{cat}\NormalTok{(}\StringTok{"}\SpecialCharTok{\textbackslash{}n}\StringTok{RENTABILIDAD DEL NUEVO MODELO}\SpecialCharTok{\textbackslash{}n}\StringTok{"}\NormalTok{, }\StringTok{"Costo anual actual:"}\NormalTok{, scales}\SpecialCharTok{::}\FunctionTok{dollar}\NormalTok{(costo\_anual\_actual), }\StringTok{"}\SpecialCharTok{\textbackslash{}n}\StringTok{"}\NormalTok{, }\StringTok{"Costo anual propuesto:"}\NormalTok{, scales}\SpecialCharTok{::}\FunctionTok{dollar}\NormalTok{(costo\_anual\_nuevo), }\StringTok{"}\SpecialCharTok{\textbackslash{}n}\StringTok{"}\NormalTok{, }\StringTok{"Ahorro anual:"}\NormalTok{, scales}\SpecialCharTok{::}\FunctionTok{dollar}\NormalTok{(ahorro\_anual), }\StringTok{"}\SpecialCharTok{\textbackslash{}n}\StringTok{"}\NormalTok{, }\StringTok{"Porcentaje de ahorro:"}\NormalTok{, }\FunctionTok{round}\NormalTok{(porcentaje\_ahorro\_anual, }\DecValTok{1}\NormalTok{), }\StringTok{"\%}\SpecialCharTok{\textbackslash{}n}\StringTok{"}\NormalTok{)}
\end{Highlighting}
\end{Shaded}

\begin{verbatim}
## 
## RENTABILIDAD DEL NUEVO MODELO
##  Costo anual actual: $112,476 
##  Costo anual propuesto: $42,178.68 
##  Ahorro anual: $70,297.79 
##  Porcentaje de ahorro: 62.5 %
\end{verbatim}

\begin{Shaded}
\begin{Highlighting}[]
\FunctionTok{cat}\NormalTok{(}\StringTok{" El modelo propuesto genera ahorros superiores al 60\%}\SpecialCharTok{\textbackslash{}n}\StringTok{"}\NormalTok{)}
\end{Highlighting}
\end{Shaded}

\begin{verbatim}
##  El modelo propuesto genera ahorros superiores al 60%
\end{verbatim}

\section{3. Los inversores desean identificar el mes más rentable y el
mes menos rentable. Esta información debe presentarse
en:}\label{los-inversores-desean-identificar-el-mes-muxe1s-rentable-y-el-mes-menos-rentable.-esta-informaciuxf3n-debe-presentarse-en}

\subsection{Una tabla comparativa
mensual.}\label{una-tabla-comparativa-mensual.}

\begin{Shaded}
\begin{Highlighting}[]
\NormalTok{tabla\_comparativa }\OtherTok{\textless{}{-}}\NormalTok{ costos\_mensuales\_nuevos }\SpecialCharTok{\%\textgreater{}\%}
  \FunctionTok{mutate}\NormalTok{(}
    \AttributeTok{Mes =}\NormalTok{ mes,}
    \StringTok{\textasciigrave{}}\AttributeTok{Precio Promedio ($/MWh)}\StringTok{\textasciigrave{}} \OtherTok{=} \FunctionTok{round}\NormalTok{(precio\_promedio, }\DecValTok{2}\NormalTok{),}
    \StringTok{\textasciigrave{}}\AttributeTok{Costo Actual ($)}\StringTok{\textasciigrave{}} \OtherTok{=} \FunctionTok{round}\NormalTok{(costo\_mensual\_actual, }\DecValTok{0}\NormalTok{)}
\NormalTok{  ) }\SpecialCharTok{\%\textgreater{}\%}
  \FunctionTok{select}\NormalTok{(Mes, }\StringTok{\textasciigrave{}}\AttributeTok{Precio Promedio ($/MWh)}\StringTok{\textasciigrave{}}\NormalTok{, }\StringTok{\textasciigrave{}}\AttributeTok{Costo Actual ($)}\StringTok{\textasciigrave{}}\NormalTok{)}

\FunctionTok{print}\NormalTok{(tabla\_comparativa)}
\end{Highlighting}
\end{Shaded}

\begin{verbatim}
## # A tibble: 12 x 3
##    Mes        `Precio Promedio ($/MWh)` `Costo Actual ($)`
##    <chr>                          <dbl>              <dbl>
##  1 ENERO                           80.7               6292
##  2 FEBRERO                         88.1               6869
##  3 MARZO                          125.                9713
##  4 ABRIL                          126.                9842
##  5 MAYO                           181.               14111
##  6 JUNIO                          184.               14351
##  7 JULIO                          122.                9529
##  8 AGOSTO                         116.                9029
##  9 SEPTIEMBRE                     104.                8087
## 10 OCTUBRE                        109.                8477
## 11 NOVIEMBRE                      107.                8358
## 12 DICIEMBRE                      100.                7819
\end{verbatim}

\subsection{Una gráfica ordenada por mes, que permita visualizar
tendencias de aumento o disminución en la
rentabilidad.}\label{una-gruxe1fica-ordenada-por-mes-que-permita-visualizar-tendencias-de-aumento-o-disminuciuxf3n-en-la-rentabilidad.}

\begin{Shaded}
\begin{Highlighting}[]
\NormalTok{datos\_grafico }\OtherTok{\textless{}{-}}\NormalTok{ costos\_mensuales\_nuevos }\SpecialCharTok{\%\textgreater{}\%}
  \FunctionTok{select}\NormalTok{(mes, mes\_num, costo\_mensual\_actual, costo\_mensual\_nuevo, precio\_promedio, ahorro\_mensual) }\SpecialCharTok{\%\textgreater{}\%}
  \FunctionTok{pivot\_longer}\NormalTok{(}\AttributeTok{cols =} \FunctionTok{c}\NormalTok{(costo\_mensual\_actual, costo\_mensual\_nuevo), }
               \AttributeTok{names\_to =} \StringTok{"modelo"}\NormalTok{, }\AttributeTok{values\_to =} \StringTok{"costo"}\NormalTok{) }\SpecialCharTok{\%\textgreater{}\%}
  \FunctionTok{mutate}\NormalTok{(}
    \AttributeTok{modelo =} \FunctionTok{case\_when}\NormalTok{(}
\NormalTok{      modelo }\SpecialCharTok{==} \StringTok{"costo\_mensual\_actual"} \SpecialCharTok{\textasciitilde{}} \StringTok{"Modelo Actual"}\NormalTok{,}
\NormalTok{      modelo }\SpecialCharTok{==} \StringTok{"costo\_mensual\_nuevo"} \SpecialCharTok{\textasciitilde{}} \StringTok{"Modelo Propuesto"}
\NormalTok{    ),}
    \AttributeTok{mes\_abrev =} \FunctionTok{factor}\NormalTok{(}\FunctionTok{substr}\NormalTok{(mes, }\DecValTok{1}\NormalTok{, }\DecValTok{3}\NormalTok{), }
                       \AttributeTok{levels =} \FunctionTok{substr}\NormalTok{(MESES, }\DecValTok{1}\NormalTok{, }\DecValTok{3}\NormalTok{))}
\NormalTok{  )}

\NormalTok{grafico\_costos }\OtherTok{\textless{}{-}} \FunctionTok{ggplot}\NormalTok{(datos\_grafico, }\FunctionTok{aes}\NormalTok{(}\AttributeTok{x =}\NormalTok{ mes\_num, }\AttributeTok{y =}\NormalTok{ costo, }\AttributeTok{color =}\NormalTok{ modelo)) }\SpecialCharTok{+}
  \FunctionTok{geom\_line}\NormalTok{(}\AttributeTok{size =} \FloatTok{1.2}\NormalTok{, }\AttributeTok{alpha =} \FloatTok{0.8}\NormalTok{) }\SpecialCharTok{+}
  \FunctionTok{geom\_point}\NormalTok{(}\AttributeTok{size =} \FloatTok{3.5}\NormalTok{, }\AttributeTok{alpha =} \FloatTok{0.9}\NormalTok{) }\SpecialCharTok{+}
  \FunctionTok{scale\_x\_continuous}\NormalTok{(}\AttributeTok{breaks =} \DecValTok{1}\SpecialCharTok{:}\DecValTok{12}\NormalTok{, }\AttributeTok{labels =} \FunctionTok{substr}\NormalTok{(MESES, }\DecValTok{1}\NormalTok{, }\DecValTok{3}\NormalTok{)) }\SpecialCharTok{+}
  \FunctionTok{scale\_y\_continuous}\NormalTok{(}\AttributeTok{labels =}\NormalTok{ scales}\SpecialCharTok{::}\FunctionTok{dollar\_format}\NormalTok{()) }\SpecialCharTok{+}
  \FunctionTok{scale\_color\_manual}\NormalTok{(}
    \AttributeTok{values =} \FunctionTok{c}\NormalTok{(}\StringTok{"Modelo Actual"} \OtherTok{=} \StringTok{"\#E74C3C"}\NormalTok{, }\StringTok{"Modelo Propuesto"} \OtherTok{=} \StringTok{"\#27AE60"}\NormalTok{),}
    \AttributeTok{name =} \StringTok{"Modelo"}
\NormalTok{  ) }\SpecialCharTok{+}
  \FunctionTok{labs}\NormalTok{(}
    \AttributeTok{title =} \StringTok{"Comparación de Costos Energéticos Mensuales 2023"}\NormalTok{,}
    \AttributeTok{subtitle =} \FunctionTok{paste}\NormalTok{(}\StringTok{"Smart Packaging {-} Ahorro anual proyectado:"}\NormalTok{, scales}\SpecialCharTok{::}\FunctionTok{dollar}\NormalTok{(ahorro\_anual)),}
    \AttributeTok{x =} \StringTok{"Mes"}\NormalTok{,}
    \AttributeTok{y =} \StringTok{"Costo Mensual (USD)"}\NormalTok{,}
    \AttributeTok{caption =} \StringTok{"Fuente: POE\_2023.xslx"}
\NormalTok{  ) }\SpecialCharTok{+}
  \FunctionTok{theme\_minimal}\NormalTok{() }\SpecialCharTok{+}
  \FunctionTok{theme}\NormalTok{(}
    \AttributeTok{plot.title =} \FunctionTok{element\_text}\NormalTok{(}\AttributeTok{size =} \DecValTok{16}\NormalTok{, }\AttributeTok{face =} \StringTok{"bold"}\NormalTok{, }\AttributeTok{hjust =} \FloatTok{0.5}\NormalTok{),}
    \AttributeTok{plot.subtitle =} \FunctionTok{element\_text}\NormalTok{(}\AttributeTok{size =} \DecValTok{12}\NormalTok{, }\AttributeTok{hjust =} \FloatTok{0.5}\NormalTok{, }\AttributeTok{color =} \StringTok{"\#349beb"}\NormalTok{),}
    \AttributeTok{legend.position =} \StringTok{"bottom"}\NormalTok{,}
    \AttributeTok{panel.grid.minor =} \FunctionTok{element\_blank}\NormalTok{(),}
    \AttributeTok{axis.text.x =} \FunctionTok{element\_text}\NormalTok{(}\AttributeTok{angle =} \DecValTok{45}\NormalTok{, }\AttributeTok{hjust =} \DecValTok{1}\NormalTok{)}
\NormalTok{  )}
\end{Highlighting}
\end{Shaded}

\begin{verbatim}
## Warning: Using `size` aesthetic for lines was deprecated in ggplot2 3.4.0.
## i Please use `linewidth` instead.
## This warning is displayed once every 8 hours.
## Call `lifecycle::last_lifecycle_warnings()` to see where this warning was
## generated.
\end{verbatim}

\begin{Shaded}
\begin{Highlighting}[]
\FunctionTok{print}\NormalTok{(grafico\_costos)}
\end{Highlighting}
\end{Shaded}

\pandocbounded{\includegraphics[keepaspectratio]{\%5BMyS1\%5DP1_G6_Copia3_files/figure-latex/unnamed-chunk-11-1.pdf}}

\begin{Shaded}
\begin{Highlighting}[]
\CommentTok{\# Filtrar los datos del mes de ENERO}
\NormalTok{enero\_data }\OtherTok{\textless{}{-}}\NormalTok{ datos }\SpecialCharTok{\%\textgreater{}\%}
  \FunctionTok{filter}\NormalTok{(mes }\SpecialCharTok{==} \StringTok{"ENERO"}\NormalTok{)}

\CommentTok{\# Función para calcular el costo total según las horas activas}
\NormalTok{calcular\_costo\_esquema }\OtherTok{\textless{}{-}} \ControlFlowTok{function}\NormalTok{(horas\_activas) \{}
\NormalTok{  enero\_data }\SpecialCharTok{\%\textgreater{}\%}
    \FunctionTok{filter}\NormalTok{(hora }\SpecialCharTok{\%in\%}\NormalTok{ horas\_activas) }\SpecialCharTok{\%\textgreater{}\%}
    \FunctionTok{summarise}\NormalTok{(}\AttributeTok{costo\_total =} \FunctionTok{sum}\NormalTok{(precio\_mwh }\SpecialCharTok{*} \FloatTok{0.2}\NormalTok{)) }\SpecialCharTok{\%\textgreater{}\%}
    \FunctionTok{pull}\NormalTok{(costo\_total)}
\NormalTok{\}}

\CommentTok{\# Definir horas activas por alternativa}
\NormalTok{horas\_a }\OtherTok{\textless{}{-}} \FunctionTok{c}\NormalTok{(}\DecValTok{4}\SpecialCharTok{:}\DecValTok{11}\NormalTok{, }\DecValTok{16}\SpecialCharTok{:}\DecValTok{23}\NormalTok{)       }\CommentTok{\# a) descansa 00{-}04 y 12{-}16}
\NormalTok{horas\_b }\OtherTok{\textless{}{-}} \FunctionTok{c}\NormalTok{(}\DecValTok{0}\SpecialCharTok{:}\DecValTok{7}\NormalTok{, }\DecValTok{12}\SpecialCharTok{:}\DecValTok{15}\NormalTok{, }\DecValTok{20}\SpecialCharTok{:}\DecValTok{23}\NormalTok{) }\CommentTok{\# b) descansa 08{-}12 y 16{-}20}
\NormalTok{horas\_c }\OtherTok{\textless{}{-}} \FunctionTok{c}\NormalTok{(}\DecValTok{0}\SpecialCharTok{:}\DecValTok{7}\NormalTok{, }\DecValTok{12}\SpecialCharTok{:}\DecValTok{19}\NormalTok{)        }\CommentTok{\# c) descansa 08{-}12 y 20{-}24}

\CommentTok{\# Calcular costos por alternativa}
\NormalTok{costo\_a }\OtherTok{\textless{}{-}} \FunctionTok{calcular\_costo\_esquema}\NormalTok{(horas\_a)}
\NormalTok{costo\_b }\OtherTok{\textless{}{-}} \FunctionTok{calcular\_costo\_esquema}\NormalTok{(horas\_b)}
\NormalTok{costo\_c }\OtherTok{\textless{}{-}} \FunctionTok{calcular\_costo\_esquema}\NormalTok{(horas\_c)}

\CommentTok{\# Crear tabla comparativa}
\NormalTok{df\_costos }\OtherTok{\textless{}{-}} \FunctionTok{data.frame}\NormalTok{(}
  \AttributeTok{Alternativa =} \FunctionTok{c}\NormalTok{(}\StringTok{"a) 00{-}04 \& 12{-}16"}\NormalTok{, }\StringTok{"b) 08{-}12 \& 16{-}20"}\NormalTok{, }\StringTok{"c) 08{-}12 \& 20{-}24"}\NormalTok{),}
  \AttributeTok{Horas\_Operativas =} \FunctionTok{c}\NormalTok{(}\FunctionTok{length}\NormalTok{(horas\_a), }\FunctionTok{length}\NormalTok{(horas\_b), }\FunctionTok{length}\NormalTok{(horas\_c)) }\SpecialCharTok{*} \DecValTok{31}\NormalTok{,  }\CommentTok{\# enero tiene 31 días}
  \AttributeTok{Consumo\_Total\_MWh =} \FunctionTok{round}\NormalTok{(}\FunctionTok{c}\NormalTok{(}\FunctionTok{length}\NormalTok{(horas\_a), }\FunctionTok{length}\NormalTok{(horas\_b), }\FunctionTok{length}\NormalTok{(horas\_c)) }\SpecialCharTok{*} \DecValTok{31} \SpecialCharTok{*} \FloatTok{0.2}\NormalTok{, }\DecValTok{2}\NormalTok{),}
  \AttributeTok{Costo\_Total\_USD =} \FunctionTok{round}\NormalTok{(}\FunctionTok{c}\NormalTok{(costo\_a, costo\_b, costo\_c), }\DecValTok{2}\NormalTok{)}
\NormalTok{)}

\CommentTok{\# Mostrar tabla}
\NormalTok{df\_costos}
\end{Highlighting}
\end{Shaded}

\begin{verbatim}
##        Alternativa Horas_Operativas Consumo_Total_MWh Costo_Total_USD
## 1 a) 00-04 & 12-16              496              99.2         7591.42
## 2 b) 08-12 & 16-20              496              99.2         7331.06
## 3 c) 08-12 & 20-24              496              99.2         8631.88
\end{verbatim}

\begin{Shaded}
\begin{Highlighting}[]
\FunctionTok{library}\NormalTok{(ggplot2)}

\FunctionTok{ggplot}\NormalTok{(df\_costos, }\FunctionTok{aes}\NormalTok{(}\AttributeTok{x =}\NormalTok{ Alternativa, }\AttributeTok{y =}\NormalTok{ Costo\_Total\_USD, }\AttributeTok{fill =}\NormalTok{ Alternativa)) }\SpecialCharTok{+}
  \FunctionTok{geom\_col}\NormalTok{(}\AttributeTok{width =} \FloatTok{0.6}\NormalTok{) }\SpecialCharTok{+}
  \FunctionTok{geom\_text}\NormalTok{(}\FunctionTok{aes}\NormalTok{(}\AttributeTok{label =} \FunctionTok{paste0}\NormalTok{(}\StringTok{"$"}\NormalTok{, }\FunctionTok{round}\NormalTok{(Costo\_Total\_USD, }\DecValTok{2}\NormalTok{))), }\AttributeTok{vjust =} \SpecialCharTok{{-}}\FloatTok{0.5}\NormalTok{) }\SpecialCharTok{+}
  \FunctionTok{labs}\NormalTok{(}
    \AttributeTok{title =} \StringTok{"Comparación de Costos por Esquema de Operación (ENERO)"}\NormalTok{,}
    \AttributeTok{x =} \StringTok{"Alternativa"}\NormalTok{,}
    \AttributeTok{y =} \StringTok{"Costo Total (USD)"}
\NormalTok{  ) }\SpecialCharTok{+}
  \FunctionTok{theme\_minimal}\NormalTok{() }\SpecialCharTok{+}
  \FunctionTok{theme}\NormalTok{(}\AttributeTok{legend.position =} \StringTok{"none"}\NormalTok{)}
\end{Highlighting}
\end{Shaded}

\pandocbounded{\includegraphics[keepaspectratio]{\%5BMyS1\%5DP1_G6_Copia3_files/figure-latex/unnamed-chunk-13-1.pdf}}
Conclusión Las tres alternativas propuestas mantuvieron el mismo consumo
total, pero el costo varió según el horario operado. La opción b) fue la
más económica, lo que demuestra que operar en horas con menor tarifa
puede reducir costos sin afectar la productividad.

\subsection{5. Enriquecimiento de la
información}\label{enriquecimiento-de-la-informaciuxf3n}

El presente análisis evalúa el consumo energético de Smart Packaging
durante el año 2023, enriqueciendo los datos originales con variables
contextuales clave para una mejor comprensión de los patrones de consumo
y su impacto en la rentabilidad del negocio.

\begin{verbatim}
## # A tibble: 1 x 7
##   total_registros consumo_promedio consumo_mediano consumo_min consumo_max
##             <int>            <dbl>           <dbl>       <dbl>       <dbl>
## 1            9855             105.            105.       0.875        453.
## # i 2 more variables: desviacion_std <dbl>, consumo_total_anual <dbl>
## # A tibble: 4 x 5
##   tipo_dia       registros consumo_promedio consumo_total desviacion_std
##   <chr>              <int>            <dbl>         <dbl>          <dbl>
## 1 Viernes             1296            114.        147252.           55.0
## 2 Semana Laboral      5508            111.        612279.           54.2
## 3 Fin de Semana       2727             93.7       255459.           45.7
## 4 Feriado              324             72.3        23413.           38.7
## # A tibble: 4 x 5
##   estacion  registros consumo_promedio consumo_total desviacion_std
##   <chr>         <int>            <dbl>         <dbl>          <dbl>
## 1 Primavera      2484            129.        319710.           49.0
## 2 Invierno       2484            125.        310279.           56.6
## 3 Otoño          2457             85.4       209819.           42.5
## 4 Verano         2430             81.7       198595.           42.0
## 
## === ANÁLISIS POR PERÍODO DEL DÍA ===
## # A tibble: 3 x 5
##   periodo_dia registros consumo_promedio consumo_total desviacion_std
##   <chr>           <int>            <dbl>         <dbl>          <dbl>
## 1 Madrugada        8395            106.        893736.           52.0
## 2 <NA>             1095            106.        115642.           58.7
## 3 Mañana            365             79.5        29024.           38.5
\end{verbatim}

\pandocbounded{\includegraphics[keepaspectratio]{\%5BMyS1\%5DP1_G6_Copia3_files/figure-latex/funciones_estadisticas-1.pdf}}
\pandocbounded{\includegraphics[keepaspectratio]{\%5BMyS1\%5DP1_G6_Copia3_files/figure-latex/funciones_estadisticas-2.pdf}}
\pandocbounded{\includegraphics[keepaspectratio]{\%5BMyS1\%5DP1_G6_Copia3_files/figure-latex/funciones_estadisticas-3.pdf}}

\begin{verbatim}
## 
## === RESUMEN DE INSIGHTS CLAVE ===
## 1. Total de registros procesados: 9855 
## 2. Consumo total anual: 1038402 MWh
## 3. Consumo promedio diario: 105.37 MWh
## 4. Archivo de datos enriquecidos guardado exitosamente
\end{verbatim}

\end{document}
